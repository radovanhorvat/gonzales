%% Generated by Sphinx.
\def\sphinxdocclass{report}
\documentclass[letterpaper,10pt,english]{sphinxmanual}
\ifdefined\pdfpxdimen
   \let\sphinxpxdimen\pdfpxdimen\else\newdimen\sphinxpxdimen
\fi \sphinxpxdimen=.75bp\relax
\ifdefined\pdfimageresolution
    \pdfimageresolution= \numexpr \dimexpr1in\relax/\sphinxpxdimen\relax
\fi
%% let collapsable pdf bookmarks panel have high depth per default
\PassOptionsToPackage{bookmarksdepth=5}{hyperref}

\PassOptionsToPackage{warn}{textcomp}
\usepackage[utf8]{inputenc}
\ifdefined\DeclareUnicodeCharacter
% support both utf8 and utf8x syntaxes
  \ifdefined\DeclareUnicodeCharacterAsOptional
    \def\sphinxDUC#1{\DeclareUnicodeCharacter{"#1}}
  \else
    \let\sphinxDUC\DeclareUnicodeCharacter
  \fi
  \sphinxDUC{00A0}{\nobreakspace}
  \sphinxDUC{2500}{\sphinxunichar{2500}}
  \sphinxDUC{2502}{\sphinxunichar{2502}}
  \sphinxDUC{2514}{\sphinxunichar{2514}}
  \sphinxDUC{251C}{\sphinxunichar{251C}}
  \sphinxDUC{2572}{\textbackslash}
\fi
\usepackage{cmap}
\usepackage[T1]{fontenc}
\usepackage{amsmath,amssymb,amstext}
\usepackage{babel}



\usepackage{tgtermes}
\usepackage{tgheros}
\renewcommand{\ttdefault}{txtt}



\usepackage[Bjarne]{fncychap}
\usepackage{sphinx}

\fvset{fontsize=auto}
\usepackage{geometry}


% Include hyperref last.
\usepackage{hyperref}
% Fix anchor placement for figures with captions.
\usepackage{hypcap}% it must be loaded after hyperref.
% Set up styles of URL: it should be placed after hyperref.
\urlstyle{same}

\addto\captionsenglish{\renewcommand{\contentsname}{Contents:}}

\usepackage{sphinxmessages}
\setcounter{tocdepth}{1}



\title{nbodysolver}
\date{Sep 10, 2021}
\release{0.1.0.dev}
\author{Radovan Horvat}
\newcommand{\sphinxlogo}{\vbox{}}
\renewcommand{\releasename}{Release}
\makeindex
\begin{document}

\pagestyle{empty}
\sphinxmaketitle
\pagestyle{plain}
\sphinxtableofcontents
\pagestyle{normal}
\phantomsection\label{\detokenize{index::doc}}



\chapter{Installation}
\label{\detokenize{install:installation}}\label{\detokenize{install::doc}}

\chapter{Quick start}
\label{\detokenize{quick_start:quick-start}}\label{\detokenize{quick_start::doc}}

\chapter{Modules}
\label{\detokenize{module_doc:modules}}\label{\detokenize{module_doc::doc}}

\section{nbody.simulator.space}
\label{\detokenize{space:nbody-simulator-space}}\label{\detokenize{space::doc}}\begin{itemize}
\item {} \index{Space (class in nbody.simulator.space)@\spxentry{Space}\spxextra{class in nbody.simulator.space}}

\begin{fulllineitems}
\phantomsection\label{\detokenize{space:nbody.simulator.space.Space}}\pysigline{\sphinxbfcode{\sphinxupquote{class }}\sphinxcode{\sphinxupquote{nbody.simulator.space.}}\sphinxbfcode{\sphinxupquote{Space}}}
\sphinxAtStartPar
Class used to represent a 3D space which contains particles.
\index{\_\_init\_\_() (nbody.simulator.space.Space method)@\spxentry{\_\_init\_\_()}\spxextra{nbody.simulator.space.Space method}}

\begin{fulllineitems}
\phantomsection\label{\detokenize{space:nbody.simulator.space.Space.__init__}}\pysiglinewithargsret{\sphinxbfcode{\sphinxupquote{\_\_init\_\_}}}{}{}
\end{fulllineitems}

\index{add\_cuboid() (nbody.simulator.space.Space method)@\spxentry{add\_cuboid()}\spxextra{nbody.simulator.space.Space method}}

\begin{fulllineitems}
\phantomsection\label{\detokenize{space:nbody.simulator.space.Space.add_cuboid}}\pysiglinewithargsret{\sphinxbfcode{\sphinxupquote{add\_cuboid}}}{\emph{\DUrole{n}{n}}, \emph{\DUrole{n}{center}}, \emph{\DUrole{n}{l\_x}}, \emph{\DUrole{n}{l\_y}}, \emph{\DUrole{n}{l\_z}}, \emph{\DUrole{n}{v\_func}}, \emph{\DUrole{n}{m\_func}}}{}
\sphinxAtStartPar
Generates uniform random particle distribution within a cuboid volume
with given center and side lengths.
\begin{quote}\begin{description}
\item[{Parameters}] \leavevmode\begin{itemize}
\item {} 
\sphinxAtStartPar
\sphinxstyleliteralstrong{\sphinxupquote{n}} \textendash{} number of particles to generate

\item {} 
\sphinxAtStartPar
\sphinxstyleliteralstrong{\sphinxupquote{center}} \textendash{} center of cuboid volume

\item {} 
\sphinxAtStartPar
\sphinxstyleliteralstrong{\sphinxupquote{l\_x}} \textendash{} length of cuboid in the x\sphinxhyphen{}direction

\item {} 
\sphinxAtStartPar
\sphinxstyleliteralstrong{\sphinxupquote{l\_y}} \textendash{} length of cuboid in the y\sphinxhyphen{}direction

\item {} 
\sphinxAtStartPar
\sphinxstyleliteralstrong{\sphinxupquote{l\_z}} \textendash{} length of cuboid in the z\sphinxhyphen{}direction

\item {} 
\sphinxAtStartPar
\sphinxstyleliteralstrong{\sphinxupquote{v\_func}} \textendash{} function of position vector

\item {} 
\sphinxAtStartPar
\sphinxstyleliteralstrong{\sphinxupquote{m\_func}} \textendash{} function of position vector

\end{itemize}

\end{description}\end{quote}

\end{fulllineitems}

\index{add\_cylinder() (nbody.simulator.space.Space method)@\spxentry{add\_cylinder()}\spxextra{nbody.simulator.space.Space method}}

\begin{fulllineitems}
\phantomsection\label{\detokenize{space:nbody.simulator.space.Space.add_cylinder}}\pysiglinewithargsret{\sphinxbfcode{\sphinxupquote{add\_cylinder}}}{\emph{\DUrole{n}{n}}, \emph{\DUrole{n}{center}}, \emph{\DUrole{n}{radius}}, \emph{\DUrole{n}{l\_z}}, \emph{\DUrole{n}{v\_func}}, \emph{\DUrole{n}{m\_func}}}{}
\sphinxAtStartPar
Generates uniform random particle distribution within a cylindrical
volume with a given center, radius and height.
\begin{quote}\begin{description}
\item[{Parameters}] \leavevmode\begin{itemize}
\item {} 
\sphinxAtStartPar
\sphinxstyleliteralstrong{\sphinxupquote{n}} \textendash{} number of particles to generate

\item {} 
\sphinxAtStartPar
\sphinxstyleliteralstrong{\sphinxupquote{center}} \textendash{} center of cylindrical volume

\item {} 
\sphinxAtStartPar
\sphinxstyleliteralstrong{\sphinxupquote{radius}} \textendash{} radius of cylindrical volume

\item {} 
\sphinxAtStartPar
\sphinxstyleliteralstrong{\sphinxupquote{l\_z}} \textendash{} height of cylindrical volume

\item {} 
\sphinxAtStartPar
\sphinxstyleliteralstrong{\sphinxupquote{v\_func}} \textendash{} function of position vector in cylindrical coordinates

\item {} 
\sphinxAtStartPar
\sphinxstyleliteralstrong{\sphinxupquote{m\_func}} \textendash{} function of position vector in cylindrical coordinates

\end{itemize}

\end{description}\end{quote}

\end{fulllineitems}

\index{add\_particle() (nbody.simulator.space.Space method)@\spxentry{add\_particle()}\spxextra{nbody.simulator.space.Space method}}

\begin{fulllineitems}
\phantomsection\label{\detokenize{space:nbody.simulator.space.Space.add_particle}}\pysiglinewithargsret{\sphinxbfcode{\sphinxupquote{add\_particle}}}{\emph{\DUrole{n}{r}}, \emph{\DUrole{n}{v}}, \emph{\DUrole{n}{m}}}{}
\sphinxAtStartPar
Add a single particle with the specified position, velocity and mass.
\begin{quote}\begin{description}
\item[{Parameters}] \leavevmode\begin{itemize}
\item {} 
\sphinxAtStartPar
\sphinxstyleliteralstrong{\sphinxupquote{r}} \textendash{} numpy array, position vector of the particle

\item {} 
\sphinxAtStartPar
\sphinxstyleliteralstrong{\sphinxupquote{v}} \textendash{} numpy array, velocity vector of the particle

\item {} 
\sphinxAtStartPar
\sphinxstyleliteralstrong{\sphinxupquote{m}} \textendash{} float, particle mass

\end{itemize}

\end{description}\end{quote}

\end{fulllineitems}

\index{add\_particles() (nbody.simulator.space.Space method)@\spxentry{add\_particles()}\spxextra{nbody.simulator.space.Space method}}

\begin{fulllineitems}
\phantomsection\label{\detokenize{space:nbody.simulator.space.Space.add_particles}}\pysiglinewithargsret{\sphinxbfcode{\sphinxupquote{add\_particles}}}{\emph{\DUrole{n}{r}}, \emph{\DUrole{n}{v}}, \emph{\DUrole{n}{m}}}{}
\sphinxAtStartPar
Adds N particles, specified by position and velocity matrices, and by mass vector.
\begin{quote}\begin{description}
\item[{Parameters}] \leavevmode\begin{itemize}
\item {} 
\sphinxAtStartPar
\sphinxstyleliteralstrong{\sphinxupquote{r}} \textendash{} N x 3 numpy array, position vectors of the particles

\item {} 
\sphinxAtStartPar
\sphinxstyleliteralstrong{\sphinxupquote{v}} \textendash{} N x 3 numpy array, velocity vectors of the particles

\item {} 
\sphinxAtStartPar
\sphinxstyleliteralstrong{\sphinxupquote{m}} \textendash{} N x 1 numpy array, mass vector

\end{itemize}

\end{description}\end{quote}

\end{fulllineitems}

\index{add\_sphere() (nbody.simulator.space.Space method)@\spxentry{add\_sphere()}\spxextra{nbody.simulator.space.Space method}}

\begin{fulllineitems}
\phantomsection\label{\detokenize{space:nbody.simulator.space.Space.add_sphere}}\pysiglinewithargsret{\sphinxbfcode{\sphinxupquote{add\_sphere}}}{\emph{\DUrole{n}{n}}, \emph{\DUrole{n}{center}}, \emph{\DUrole{n}{radius}}, \emph{\DUrole{n}{v\_func}}, \emph{\DUrole{n}{m\_func}}}{}
\sphinxAtStartPar
Generates uniform random particle distribution within a spherical volume
with a given center and radius.
\begin{quote}\begin{description}
\item[{Parameters}] \leavevmode\begin{itemize}
\item {} 
\sphinxAtStartPar
\sphinxstyleliteralstrong{\sphinxupquote{n}} \textendash{} number of particles to generate

\item {} 
\sphinxAtStartPar
\sphinxstyleliteralstrong{\sphinxupquote{center}} \textendash{} center of spherical volume

\item {} 
\sphinxAtStartPar
\sphinxstyleliteralstrong{\sphinxupquote{radius}} \textendash{} radius of spherical volume

\item {} 
\sphinxAtStartPar
\sphinxstyleliteralstrong{\sphinxupquote{v\_func}} \textendash{} function of r, theta, phi

\item {} 
\sphinxAtStartPar
\sphinxstyleliteralstrong{\sphinxupquote{m\_func}} \textendash{} function of r, theta, phi

\end{itemize}

\end{description}\end{quote}

\end{fulllineitems}

\index{clear\_particles() (nbody.simulator.space.Space method)@\spxentry{clear\_particles()}\spxextra{nbody.simulator.space.Space method}}

\begin{fulllineitems}
\phantomsection\label{\detokenize{space:nbody.simulator.space.Space.clear_particles}}\pysiglinewithargsret{\sphinxbfcode{\sphinxupquote{clear\_particles}}}{}{}
\sphinxAtStartPar
Removes all particles from the space.

\end{fulllineitems}

\index{from\_hdf5() (nbody.simulator.space.Space method)@\spxentry{from\_hdf5()}\spxextra{nbody.simulator.space.Space method}}

\begin{fulllineitems}
\phantomsection\label{\detokenize{space:nbody.simulator.space.Space.from_hdf5}}\pysiglinewithargsret{\sphinxbfcode{\sphinxupquote{from\_hdf5}}}{\emph{\DUrole{n}{filepath}}}{}
\sphinxAtStartPar
Reads data from hdf5 file.

\end{fulllineitems}

\index{to\_hdf5() (nbody.simulator.space.Space method)@\spxentry{to\_hdf5()}\spxextra{nbody.simulator.space.Space method}}

\begin{fulllineitems}
\phantomsection\label{\detokenize{space:nbody.simulator.space.Space.to_hdf5}}\pysiglinewithargsret{\sphinxbfcode{\sphinxupquote{to\_hdf5}}}{\emph{\DUrole{n}{filepath}}}{}
\sphinxAtStartPar
Saves configuration to hdf5 file. Datasets are called “r”, “v” and “m”.

\end{fulllineitems}


\end{fulllineitems}


\end{itemize}


\section{nbody.simulator.simulation}
\label{\detokenize{simulator:nbody-simulator-simulation}}\label{\detokenize{simulator::doc}}\begin{itemize}
\item {} \index{SimulationBase (class in nbody.simulator.simulation)@\spxentry{SimulationBase}\spxextra{class in nbody.simulator.simulation}}

\begin{fulllineitems}
\phantomsection\label{\detokenize{simulator:nbody.simulator.simulation.SimulationBase}}\pysiglinewithargsret{\sphinxbfcode{\sphinxupquote{class }}\sphinxcode{\sphinxupquote{nbody.simulator.simulation.}}\sphinxbfcode{\sphinxupquote{SimulationBase}}}{\emph{\DUrole{n}{space}}, \emph{\DUrole{n}{output\_filepath}}, \emph{\DUrole{n}{G}}, \emph{\DUrole{n}{eps}}}{}~\index{\_\_init\_\_() (nbody.simulator.simulation.SimulationBase method)@\spxentry{\_\_init\_\_()}\spxextra{nbody.simulator.simulation.SimulationBase method}}

\begin{fulllineitems}
\phantomsection\label{\detokenize{simulator:nbody.simulator.simulation.SimulationBase.__init__}}\pysiglinewithargsret{\sphinxbfcode{\sphinxupquote{\_\_init\_\_}}}{\emph{\DUrole{n}{space}}, \emph{\DUrole{n}{output\_filepath}}, \emph{\DUrole{n}{G}}, \emph{\DUrole{n}{eps}}}{}
\sphinxAtStartPar
Base class for all simulation types.
\begin{quote}\begin{description}
\item[{Parameters}] \leavevmode\begin{itemize}
\item {} 
\sphinxAtStartPar
\sphinxstyleliteralstrong{\sphinxupquote{space}} \textendash{} instance of Space

\item {} 
\sphinxAtStartPar
\sphinxstyleliteralstrong{\sphinxupquote{output\_filepath}} \textendash{} output filepath

\item {} 
\sphinxAtStartPar
\sphinxstyleliteralstrong{\sphinxupquote{G}} \textendash{} gravitational constant

\item {} 
\sphinxAtStartPar
\sphinxstyleliteralstrong{\sphinxupquote{eps}} \textendash{} gravitational softening

\end{itemize}

\end{description}\end{quote}

\end{fulllineitems}

\index{add\_result() (nbody.simulator.simulation.SimulationBase method)@\spxentry{add\_result()}\spxextra{nbody.simulator.simulation.SimulationBase method}}

\begin{fulllineitems}
\phantomsection\label{\detokenize{simulator:nbody.simulator.simulation.SimulationBase.add_result}}\pysiglinewithargsret{\sphinxbfcode{\sphinxupquote{add\_result}}}{\emph{\DUrole{n}{res\_name}}, \emph{\DUrole{n}{res\_frequency}\DUrole{o}{=}\DUrole{default_value}{1}}}{}
\sphinxAtStartPar
Adds a result which shall be written to the hdf5 file during the simulation. The shape
of the result needs to be provided, as does the frequency of the output. The default res\_frequency
1 means the result will be written to the file at each simulation step.
\begin{quote}\begin{description}
\item[{Parameters}] \leavevmode\begin{itemize}
\item {} 
\sphinxAtStartPar
\sphinxstyleliteralstrong{\sphinxupquote{res\_name}} \textendash{} string, name of result

\item {} 
\sphinxAtStartPar
\sphinxstyleliteralstrong{\sphinxupquote{res\_frequency}} \textendash{} int, result will be written every res\_frequency steps. If 0, result will not
be written at all.

\end{itemize}

\item[{Returns}] \leavevmode
\sphinxAtStartPar


\end{description}\end{quote}

\end{fulllineitems}

\index{create\_datasets() (nbody.simulator.simulation.SimulationBase method)@\spxentry{create\_datasets()}\spxextra{nbody.simulator.simulation.SimulationBase method}}

\begin{fulllineitems}
\phantomsection\label{\detokenize{simulator:nbody.simulator.simulation.SimulationBase.create_datasets}}\pysiglinewithargsret{\sphinxbfcode{\sphinxupquote{create\_datasets}}}{\emph{\DUrole{n}{hdf5\_fobj}}, \emph{\DUrole{n}{n\_steps}}, \emph{\DUrole{n}{step\_size}}}{}
\sphinxAtStartPar
Creates hdf5 datasets.

\end{fulllineitems}

\index{run() (nbody.simulator.simulation.SimulationBase method)@\spxentry{run()}\spxextra{nbody.simulator.simulation.SimulationBase method}}

\begin{fulllineitems}
\phantomsection\label{\detokenize{simulator:nbody.simulator.simulation.SimulationBase.run}}\pysiglinewithargsret{\sphinxbfcode{\sphinxupquote{run}}}{\emph{\DUrole{n}{n\_steps}}, \emph{\DUrole{n}{step\_size}}}{}
\sphinxAtStartPar
Runs the simulation.
\begin{quote}\begin{description}
\item[{Parameters}] \leavevmode\begin{itemize}
\item {} 
\sphinxAtStartPar
\sphinxstyleliteralstrong{\sphinxupquote{n\_steps}} \textendash{} number of steps

\item {} 
\sphinxAtStartPar
\sphinxstyleliteralstrong{\sphinxupquote{step\_size}} \textendash{} step size

\end{itemize}

\end{description}\end{quote}

\end{fulllineitems}

\index{set\_kernel() (nbody.simulator.simulation.SimulationBase method)@\spxentry{set\_kernel()}\spxextra{nbody.simulator.simulation.SimulationBase method}}

\begin{fulllineitems}
\phantomsection\label{\detokenize{simulator:nbody.simulator.simulation.SimulationBase.set_kernel}}\pysiglinewithargsret{\sphinxbfcode{\sphinxupquote{set\_kernel}}}{\emph{\DUrole{n}{kernel\_func}}}{}
\sphinxAtStartPar
Sets the kernel function which will be used to calculate accelerations for each simulation
step.
\begin{quote}\begin{description}
\item[{Parameters}] \leavevmode
\sphinxAtStartPar
\sphinxstyleliteralstrong{\sphinxupquote{kernel\_func}} \textendash{} function used to calculate accelerations

\end{description}\end{quote}

\end{fulllineitems}

\index{set\_metadata() (nbody.simulator.simulation.SimulationBase static method)@\spxentry{set\_metadata()}\spxextra{nbody.simulator.simulation.SimulationBase static method}}

\begin{fulllineitems}
\phantomsection\label{\detokenize{simulator:nbody.simulator.simulation.SimulationBase.set_metadata}}\pysiglinewithargsret{\sphinxbfcode{\sphinxupquote{static }}\sphinxbfcode{\sphinxupquote{set\_metadata}}}{\emph{\DUrole{n}{hdf5\_obj}}, \emph{\DUrole{o}{**}\DUrole{n}{kwargs}}}{}
\sphinxAtStartPar
Sets metadata attributes for hdf5 object (can be a group or a dataset)

\end{fulllineitems}


\end{fulllineitems}


\end{itemize}

\begin{DUlineblock}{0em}
\item[] 
\end{DUlineblock}
\begin{itemize}
\item {} \index{PPSimulation (class in nbody.simulator.simulation)@\spxentry{PPSimulation}\spxextra{class in nbody.simulator.simulation}}

\begin{fulllineitems}
\phantomsection\label{\detokenize{simulator:nbody.simulator.simulation.PPSimulation}}\pysiglinewithargsret{\sphinxbfcode{\sphinxupquote{class }}\sphinxcode{\sphinxupquote{nbody.simulator.simulation.}}\sphinxbfcode{\sphinxupquote{PPSimulation}}}{\emph{\DUrole{o}{*}\DUrole{n}{args}}, \emph{\DUrole{o}{**}\DUrole{n}{kwargs}}}{}
\sphinxAtStartPar
Bases: {\hyperref[\detokenize{simulator:nbody.simulator.simulation.SimulationBase}]{\sphinxcrossref{\sphinxcode{\sphinxupquote{nbody.simulator.simulation.SimulationBase}}}}}

\sphinxAtStartPar
Simulation class for brute\sphinxhyphen{}force simulations.
\index{\_\_init\_\_() (nbody.simulator.simulation.PPSimulation method)@\spxentry{\_\_init\_\_()}\spxextra{nbody.simulator.simulation.PPSimulation method}}

\begin{fulllineitems}
\phantomsection\label{\detokenize{simulator:nbody.simulator.simulation.PPSimulation.__init__}}\pysiglinewithargsret{\sphinxbfcode{\sphinxupquote{\_\_init\_\_}}}{\emph{\DUrole{o}{*}\DUrole{n}{args}}, \emph{\DUrole{o}{**}\DUrole{n}{kwargs}}}{}~\begin{quote}\begin{description}
\item[{Parameters}] \leavevmode\begin{itemize}
\item {} 
\sphinxAtStartPar
\sphinxstyleliteralstrong{\sphinxupquote{space}} \textendash{} instance of Space

\item {} 
\sphinxAtStartPar
\sphinxstyleliteralstrong{\sphinxupquote{output\_filepath}} \textendash{} output filepath

\item {} 
\sphinxAtStartPar
\sphinxstyleliteralstrong{\sphinxupquote{G}} \textendash{} gravitational constant

\item {} 
\sphinxAtStartPar
\sphinxstyleliteralstrong{\sphinxupquote{eps}} \textendash{} gravitational softening

\end{itemize}

\end{description}\end{quote}

\end{fulllineitems}

\index{calc\_accs() (nbody.simulator.simulation.PPSimulation method)@\spxentry{calc\_accs()}\spxextra{nbody.simulator.simulation.PPSimulation method}}

\begin{fulllineitems}
\phantomsection\label{\detokenize{simulator:nbody.simulator.simulation.PPSimulation.calc_accs}}\pysiglinewithargsret{\sphinxbfcode{\sphinxupquote{calc\_accs}}}{}{}
\sphinxAtStartPar
Calculates the acceleration vector, using the kernel which was set.

\end{fulllineitems}


\end{fulllineitems}


\end{itemize}

\begin{DUlineblock}{0em}
\item[] 
\end{DUlineblock}
\begin{itemize}
\item {} \index{BHSimulation (class in nbody.simulator.simulation)@\spxentry{BHSimulation}\spxextra{class in nbody.simulator.simulation}}

\begin{fulllineitems}
\phantomsection\label{\detokenize{simulator:nbody.simulator.simulation.BHSimulation}}\pysiglinewithargsret{\sphinxbfcode{\sphinxupquote{class }}\sphinxcode{\sphinxupquote{nbody.simulator.simulation.}}\sphinxbfcode{\sphinxupquote{BHSimulation}}}{\emph{\DUrole{n}{space}}, \emph{\DUrole{n}{output\_filepath}}, \emph{\DUrole{n}{G}}, \emph{\DUrole{n}{eps}}, \emph{\DUrole{n}{root\_width}}, \emph{\DUrole{n}{root\_center}}, \emph{\DUrole{n}{theta}}}{}
\sphinxAtStartPar
Bases: {\hyperref[\detokenize{simulator:nbody.simulator.simulation.SimulationBase}]{\sphinxcrossref{\sphinxcode{\sphinxupquote{nbody.simulator.simulation.SimulationBase}}}}}

\sphinxAtStartPar
Simulation class for Barnes\sphinxhyphen{}Hut simulations.
\index{\_\_init\_\_() (nbody.simulator.simulation.BHSimulation method)@\spxentry{\_\_init\_\_()}\spxextra{nbody.simulator.simulation.BHSimulation method}}

\begin{fulllineitems}
\phantomsection\label{\detokenize{simulator:nbody.simulator.simulation.BHSimulation.__init__}}\pysiglinewithargsret{\sphinxbfcode{\sphinxupquote{\_\_init\_\_}}}{\emph{\DUrole{n}{space}}, \emph{\DUrole{n}{output\_filepath}}, \emph{\DUrole{n}{G}}, \emph{\DUrole{n}{eps}}, \emph{\DUrole{n}{root\_width}}, \emph{\DUrole{n}{root\_center}}, \emph{\DUrole{n}{theta}}}{}~\begin{quote}\begin{description}
\item[{Parameters}] \leavevmode\begin{itemize}
\item {} 
\sphinxAtStartPar
\sphinxstyleliteralstrong{\sphinxupquote{space}} \textendash{} instance of Space

\item {} 
\sphinxAtStartPar
\sphinxstyleliteralstrong{\sphinxupquote{output\_filepath}} \textendash{} output filepath

\item {} 
\sphinxAtStartPar
\sphinxstyleliteralstrong{\sphinxupquote{G}} \textendash{} gravitational constant

\item {} 
\sphinxAtStartPar
\sphinxstyleliteralstrong{\sphinxupquote{eps}} \textendash{} gravitational softening

\item {} 
\sphinxAtStartPar
\sphinxstyleliteralstrong{\sphinxupquote{root\_width}} \textendash{} octree root node width

\item {} 
\sphinxAtStartPar
\sphinxstyleliteralstrong{\sphinxupquote{root\_center}} \textendash{} octree root node center

\item {} 
\sphinxAtStartPar
\sphinxstyleliteralstrong{\sphinxupquote{theta}} \textendash{} threshold parameter

\end{itemize}

\end{description}\end{quote}

\end{fulllineitems}

\index{calc\_accs() (nbody.simulator.simulation.BHSimulation method)@\spxentry{calc\_accs()}\spxextra{nbody.simulator.simulation.BHSimulation method}}

\begin{fulllineitems}
\phantomsection\label{\detokenize{simulator:nbody.simulator.simulation.BHSimulation.calc_accs}}\pysiglinewithargsret{\sphinxbfcode{\sphinxupquote{calc\_accs}}}{}{}
\sphinxAtStartPar
Calculates the acceleration vector, using the kernel which was set.

\end{fulllineitems}


\end{fulllineitems}


\end{itemize}


\section{nbody.lib.brute\_force}
\label{\detokenize{brute_force:nbody-lib-brute-force}}\label{\detokenize{brute_force::doc}}\begin{itemize}
\item {} \index{calculate\_accs\_pp() (in module nbody.lib.brute\_force)@\spxentry{calculate\_accs\_pp()}\spxextra{in module nbody.lib.brute\_force}}

\begin{fulllineitems}
\phantomsection\label{\detokenize{brute_force:nbody.lib.brute_force.calculate_accs_pp}}\pysiglinewithargsret{\sphinxcode{\sphinxupquote{nbody.lib.brute\_force.}}\sphinxbfcode{\sphinxupquote{calculate\_accs\_pp}}}{\emph{\DUrole{n}{r}}, \emph{\DUrole{n}{m}}, \emph{\DUrole{n}{G}}, \emph{\DUrole{n}{eps}}}{}
\sphinxAtStartPar
Parallel brute\sphinxhyphen{}force calculation of accelerations.
\begin{quote}\begin{description}
\item[{Parameters}] \leavevmode\begin{itemize}
\item {} 
\sphinxAtStartPar
\sphinxstyleliteralstrong{\sphinxupquote{r}} \textendash{} N x 3 numpy array, position vectors of the particles

\item {} 
\sphinxAtStartPar
\sphinxstyleliteralstrong{\sphinxupquote{m}} \textendash{} N x 1 numpy array, mass vector of the particles

\item {} 
\sphinxAtStartPar
\sphinxstyleliteralstrong{\sphinxupquote{G}} \textendash{} gravitational constant

\item {} 
\sphinxAtStartPar
\sphinxstyleliteralstrong{\sphinxupquote{eps}} \textendash{} gravitational softening

\end{itemize}

\item[{Returns}] \leavevmode
\sphinxAtStartPar
N x 3 numpy array, acceleration vectors of particles

\end{description}\end{quote}

\end{fulllineitems}


\end{itemize}


\section{nbody.lib.octree}
\label{\detokenize{octree:nbody-lib-octree}}\label{\detokenize{octree::doc}}\begin{itemize}
\item {} \index{calc\_accs\_octree() (in module nbody.lib.octree)@\spxentry{calc\_accs\_octree()}\spxextra{in module nbody.lib.octree}}

\begin{fulllineitems}
\phantomsection\label{\detokenize{octree:nbody.lib.octree.calc_accs_octree}}\pysiglinewithargsret{\sphinxcode{\sphinxupquote{nbody.lib.octree.}}\sphinxbfcode{\sphinxupquote{calc\_accs\_octree}}}{\emph{\DUrole{n}{w}}, \emph{\DUrole{n}{r\_x}}, \emph{\DUrole{n}{r\_y}}, \emph{\DUrole{n}{r\_z}}, \emph{\DUrole{n}{r}}, \emph{\DUrole{n}{m}}, \emph{\DUrole{n}{G}}, \emph{\DUrole{n}{eps}}, \emph{\DUrole{n}{theta}}}{}~\begin{description}
\item[{Parallel calculation of accelerations using the Barnes\sphinxhyphen{}Hut algorithm. Constructs the}] \leavevmode
\sphinxAtStartPar
octree and calculates the accelerations.
\begin{quote}\begin{description}
\item[{param w}] \leavevmode
\sphinxAtStartPar
double, root node width

\item[{param r\_x}] \leavevmode
\sphinxAtStartPar
double, x\sphinxhyphen{}coordinate of root node center

\item[{param r\_y}] \leavevmode
\sphinxAtStartPar
double, y\sphinxhyphen{}coordinate of root node center

\item[{param r\_z}] \leavevmode
\sphinxAtStartPar
double, z\sphinxhyphen{}coordinate of root node center

\item[{param r}] \leavevmode
\sphinxAtStartPar
N x 3 numpy array, position vectors of the particles

\item[{param m}] \leavevmode
\sphinxAtStartPar
N x 1 numpy array, mass vector of the particles

\item[{param G}] \leavevmode
\sphinxAtStartPar
gravitational constant

\item[{param eps}] \leavevmode
\sphinxAtStartPar
gravitational softening

\item[{param theta}] \leavevmode
\sphinxAtStartPar
threshold parameter

\item[{return}] \leavevmode
\sphinxAtStartPar
N x 3 numpy array, acceleration vectors of particles

\end{description}\end{quote}

\end{description}

\end{fulllineitems}


\end{itemize}


\section{nbody.lib.physics}
\label{\detokenize{physics:nbody-lib-physics}}\label{\detokenize{physics::doc}}\begin{itemize}
\item {} \index{calc\_com() (in module nbody.lib.physics)@\spxentry{calc\_com()}\spxextra{in module nbody.lib.physics}}

\begin{fulllineitems}
\phantomsection\label{\detokenize{physics:nbody.lib.physics.calc_com}}\pysiglinewithargsret{\sphinxcode{\sphinxupquote{nbody.lib.physics.}}\sphinxbfcode{\sphinxupquote{calc\_com}}}{\emph{\DUrole{n}{r}}, \emph{\DUrole{n}{m}}}{}
\sphinxAtStartPar
Calculates center of mass of a set of particles.
\begin{quote}\begin{description}
\item[{Parameters}] \leavevmode\begin{itemize}
\item {} 
\sphinxAtStartPar
\sphinxstyleliteralstrong{\sphinxupquote{r}} \textendash{} N x 3 numpy array, position vectors of the particles

\item {} 
\sphinxAtStartPar
\sphinxstyleliteralstrong{\sphinxupquote{m}} \textendash{} N x 1 numpy array, mass vector of the particles

\end{itemize}

\item[{Returns}] \leavevmode
\sphinxAtStartPar
1 x 3 numpy array, center of mass

\end{description}\end{quote}

\end{fulllineitems}


\end{itemize}

\begin{DUlineblock}{0em}
\item[] 
\end{DUlineblock}
\begin{itemize}
\item {} \index{calc\_ke() (in module nbody.lib.physics)@\spxentry{calc\_ke()}\spxextra{in module nbody.lib.physics}}

\begin{fulllineitems}
\phantomsection\label{\detokenize{physics:nbody.lib.physics.calc_ke}}\pysiglinewithargsret{\sphinxcode{\sphinxupquote{nbody.lib.physics.}}\sphinxbfcode{\sphinxupquote{calc\_ke}}}{\emph{\DUrole{n}{v}}, \emph{\DUrole{n}{m}}}{}
\sphinxAtStartPar
Calculates kinetic energy of a system of particles.
\begin{quote}\begin{description}
\item[{Parameters}] \leavevmode\begin{itemize}
\item {} 
\sphinxAtStartPar
\sphinxstyleliteralstrong{\sphinxupquote{v}} \textendash{} N x 3 numpy array, velocity vectors of the particles

\item {} 
\sphinxAtStartPar
\sphinxstyleliteralstrong{\sphinxupquote{m}} \textendash{} N x 1 numpy array, mass vector of the particles

\end{itemize}

\item[{Returns}] \leavevmode
\sphinxAtStartPar
double, kinetic energy of the system

\end{description}\end{quote}

\end{fulllineitems}


\end{itemize}

\begin{DUlineblock}{0em}
\item[] 
\end{DUlineblock}
\begin{itemize}
\item {} \index{calc\_pe() (in module nbody.lib.physics)@\spxentry{calc\_pe()}\spxextra{in module nbody.lib.physics}}

\begin{fulllineitems}
\phantomsection\label{\detokenize{physics:nbody.lib.physics.calc_pe}}\pysiglinewithargsret{\sphinxcode{\sphinxupquote{nbody.lib.physics.}}\sphinxbfcode{\sphinxupquote{calc\_pe}}}{\emph{\DUrole{n}{r}}, \emph{\DUrole{n}{m}}, \emph{\DUrole{n}{G}}, \emph{\DUrole{n}{eps}}}{}
\sphinxAtStartPar
Calculates potential energy of a system of particles.
\begin{quote}\begin{description}
\item[{Parameters}] \leavevmode\begin{itemize}
\item {} 
\sphinxAtStartPar
\sphinxstyleliteralstrong{\sphinxupquote{r}} \textendash{} N x 3 numpy array, position vectors of the particles

\item {} 
\sphinxAtStartPar
\sphinxstyleliteralstrong{\sphinxupquote{m}} \textendash{} N x 1 numpy array, mass vector of the particles

\item {} 
\sphinxAtStartPar
\sphinxstyleliteralstrong{\sphinxupquote{G}} \textendash{} gravitational constant

\item {} 
\sphinxAtStartPar
\sphinxstyleliteralstrong{\sphinxupquote{eps}} \textendash{} gravitational softening

\end{itemize}

\item[{Returns}] \leavevmode
\sphinxAtStartPar
double, potential energy of the system

\end{description}\end{quote}

\end{fulllineitems}


\end{itemize}

\begin{DUlineblock}{0em}
\item[] 
\end{DUlineblock}
\begin{itemize}
\item {} \index{calc\_te() (in module nbody.lib.physics)@\spxentry{calc\_te()}\spxextra{in module nbody.lib.physics}}

\begin{fulllineitems}
\phantomsection\label{\detokenize{physics:nbody.lib.physics.calc_te}}\pysiglinewithargsret{\sphinxcode{\sphinxupquote{nbody.lib.physics.}}\sphinxbfcode{\sphinxupquote{calc\_te}}}{\emph{\DUrole{n}{r}}, \emph{\DUrole{n}{v}}, \emph{\DUrole{n}{m}}, \emph{\DUrole{n}{G}}, \emph{\DUrole{n}{eps}}}{}
\sphinxAtStartPar
Calculates total mechanical energy of a system of particles.
\begin{quote}\begin{description}
\item[{Parameters}] \leavevmode\begin{itemize}
\item {} 
\sphinxAtStartPar
\sphinxstyleliteralstrong{\sphinxupquote{r}} \textendash{} N x 3 numpy array, position vectors of the particles

\item {} 
\sphinxAtStartPar
\sphinxstyleliteralstrong{\sphinxupquote{v}} \textendash{} N x 3 numpy array, velocity vectors of the particles

\item {} 
\sphinxAtStartPar
\sphinxstyleliteralstrong{\sphinxupquote{m}} \textendash{} N x 1 numpy array, mass vector of the particles

\item {} 
\sphinxAtStartPar
\sphinxstyleliteralstrong{\sphinxupquote{G}} \textendash{} gravitational constant

\item {} 
\sphinxAtStartPar
\sphinxstyleliteralstrong{\sphinxupquote{eps}} \textendash{} gravitational softening

\end{itemize}

\item[{Returns}] \leavevmode
\sphinxAtStartPar
double, total mechanical energy of the system

\end{description}\end{quote}

\end{fulllineitems}


\end{itemize}

\begin{DUlineblock}{0em}
\item[] 
\end{DUlineblock}
\begin{itemize}
\item {} \index{calc\_ang\_mom() (in module nbody.lib.physics)@\spxentry{calc\_ang\_mom()}\spxextra{in module nbody.lib.physics}}

\begin{fulllineitems}
\phantomsection\label{\detokenize{physics:nbody.lib.physics.calc_ang_mom}}\pysiglinewithargsret{\sphinxcode{\sphinxupquote{nbody.lib.physics.}}\sphinxbfcode{\sphinxupquote{calc\_ang\_mom}}}{\emph{\DUrole{n}{r}}, \emph{\DUrole{n}{v}}, \emph{\DUrole{n}{m}}}{}
\sphinxAtStartPar
Calculates angular momentum of a system of particles.
\begin{quote}\begin{description}
\item[{Parameters}] \leavevmode\begin{itemize}
\item {} 
\sphinxAtStartPar
\sphinxstyleliteralstrong{\sphinxupquote{r}} \textendash{} N x 3 numpy array, position vectors of the particles

\item {} 
\sphinxAtStartPar
\sphinxstyleliteralstrong{\sphinxupquote{v}} \textendash{} N x 3 numpy array, velocity vectors of the particles

\item {} 
\sphinxAtStartPar
\sphinxstyleliteralstrong{\sphinxupquote{m}} \textendash{} N x 1 numpy array, mass vector of the particles

\end{itemize}

\item[{Returns}] \leavevmode
\sphinxAtStartPar
1 x 3 numpy array, angular momentum vector

\end{description}\end{quote}

\end{fulllineitems}


\end{itemize}


\section{nbody.ui.viewer}
\label{\detokenize{viewer:nbody-ui-viewer}}\label{\detokenize{viewer::doc}}\begin{itemize}
\item {} \index{run\_viewer() (in module nbody.ui.viewer)@\spxentry{run\_viewer()}\spxextra{in module nbody.ui.viewer}}

\begin{fulllineitems}
\phantomsection\label{\detokenize{viewer:nbody.ui.viewer.run_viewer}}\pysiglinewithargsret{\sphinxcode{\sphinxupquote{nbody.ui.viewer.}}\sphinxbfcode{\sphinxupquote{run\_viewer}}}{\emph{\DUrole{n}{filename}\DUrole{o}{=}\DUrole{default_value}{\textquotesingle{}\textquotesingle{}}}, \emph{\DUrole{n}{body\_sizes}\DUrole{o}{=}\DUrole{default_value}{4}}, \emph{\DUrole{n}{colors}\DUrole{o}{=}\DUrole{default_value}{(0.9, 0.9, 0.1, 0.7)}}}{}
\sphinxAtStartPar
Runs the viewer with given filename.
\begin{quote}\begin{description}
\item[{Parameters}] \leavevmode\begin{itemize}
\item {} 
\sphinxAtStartPar
\sphinxstyleliteralstrong{\sphinxupquote{filename}} \textendash{} filepath of hdf5 file

\item {} 
\sphinxAtStartPar
\sphinxstyleliteralstrong{\sphinxupquote{body\_sizes}} \textendash{} int or tuple of ints, respective body size(s) which will be drawn.

\item {} 
\sphinxAtStartPar
\sphinxstyleliteralstrong{\sphinxupquote{colors}} \textendash{} double or tuple of doubles, respective body color(s) which will be drawn.

\end{itemize}

\end{description}\end{quote}

\end{fulllineitems}


\end{itemize}


\chapter{Indices and tables}
\label{\detokenize{index:indices-and-tables}}\begin{itemize}
\item {} 
\sphinxAtStartPar
\DUrole{xref,std,std-ref}{genindex}

\item {} 
\sphinxAtStartPar
\DUrole{xref,std,std-ref}{modindex}

\item {} 
\sphinxAtStartPar
\DUrole{xref,std,std-ref}{search}

\end{itemize}



\renewcommand{\indexname}{Index}
\printindex
\end{document}